\markboth{序}{序} \vspace*{0.0cm}
\thispagestyle{empty}
\vspace*{0.5cm}
\centerline{\zihao{-2}\hei{\color{darkblue}{序}}}\vspace{40pt}

在全国大学生数学建模的活动中,我认识了吕同富老师.在多次交往中,我发现吕老师是个非常勤奋的人,闲暇时,他总是在编辑自己的图书.去年4月,他送我一本他的书《数值计算方法》,我翻翻目录,内容正是学生实践活动中对建立的数学模型需要应用数值方法和数学软件,尤其MATLAB进行验证时所需要的,正好成为我带领学生们学习其中的内容并利用这些知识解决实际问题的教材和参考书.通过使用,我和研究生们对这本书有共同的印象:不仅内容翔实、全面,而且实用;既有数学理论的抽象性和严谨性,又有实用性和实验性的技术特征,是一本理论性和实践性都很强的书籍.去年9月,他给我这本书的第2 版电子版本,希望我看后能为本书写序.坦率地说我很犹豫,这本书,尽管我们使用起来非常好,但我毕竟不是这方面的专家,序写得不好会给本书带来不好的影响,所以一直拖着.但我无论如何也无法拒绝吕老师的请求,相反我很想为他和他的书叫好.一个在高校很有影响的教授,放弃休息时间,执着地著书立说、传播知识,对这样一位教授的请求,似乎谁也没有理由拒绝!

我是从事偏微分方程理论研究的,由于近十年参加数学建模的培训、指导,越来越喜欢数学建模,逐步地将数学建模与自己的科学研究结合起来,越来越感到科学研究仅有纯粹的理论研究部分是不够的,完成理论分析后必须进行数值计算和模拟,这样才算是一个完整的研究过程。实际上,现代科学都在走向定量化和精确化,并形成了一系列计算性的学科分支,如计算物理、计算化学、计算生物学、计算地质学、计算气象学和计算材料学等.计算能力是计算工具和计算方法的效率的乘积,提高计算方法的效率与提高计算机硬件的效率同样重要.数值计算方法是一种研究和求解数学问题数值近似解的方法, 是在计算机上使用的解数学问题的方法.在科学研究和工程技术中要用到各种计算方法. 例如:在航空航天、地质勘探、汽车制造、桥梁设计、 天气预报和汉字字样设计等都有数值计算方法的应用,科学计算已用到科学技术和社会生活的各个领域中.如今,随着计算机技术的迅速发展和普及,数值计算方法课程已经成为所有理工科和经济管理学科学生的必修课程.

随着计算机的不断发展和进步,优秀的数学软件MATLAB 应运而生,MATLAB一问世就以它强大的功能,被广大科技工作者公认为科学计算最好的软件之一.为使数值计算方法与计算机更好地结合,吕同富教授编写了《数值计算方法》.一年来,我带领研究生们应用本书,发现这本书与国内外同类教材相比有以下特点:

(1)内容相对系统完整而简捷,精练的论述几乎涵盖了经典数值分析的全部内容:包括非线性方程的数值解法;线性方程组的数值解法;矩阵特征值与特征向量的数值算法;插值方法;函数逼近;数值积分;数值微分;常微分方程数值解等.

(2)本书重点讲述了数值计算方法的思想和原理,尽可能避免了过深的数学理论和过于繁杂的算法细节的描述,便于理解、阅读与应用.

(3)本书对所有经典的数值计算方法都给出了MATLAB程序,这在国内外同类教材中是少见的,也是本书的亮点.这不仅有助于读者利用MATLAB 的超强功能解决科学计算问题,更有助于避免那种学过数值计算方法但不能上机解决实际问题的现象发生.


以上就是我对吕老师的看法以及我和同学们对本书的看法,姑且当作本书第2版的序,相信同仁们看了用了本书后一定赞成我的观点.


\vspace{2cm}

\hfill 谭忠\hspace{1.5em}

\hfill 厦门大学\hspace{0.5em}

\hfill 2013年6月\hspace{0em}
