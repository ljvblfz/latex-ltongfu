%-------------------------------------宏包引用---------------------------------------------------
\usepackage[paperwidth=185mm,paperheight=230mm,textheight=190mm,textwidth=145mm,left=20mm,right=20mm, top=25mm, bottom=15mm]{geometry}            %定义版面
%--------------------------------------------------------------------------------------
\usepackage{fontspec}
\usepackage{xunicode}
\usepackage{xltxtra}
%--------------------------------------------------------------------------------------
%\usepackage{fancyhdr}                 % 页眉和页脚的相关定义
\usepackage[CJKbookmarks, colorlinks, bookmarksnumbered=true,pdfstartview=FitH,linkcolor=black]{hyperref}   % 书签功能,选项去掉链接红色方框
%-----------------------------------------------------------------------------------------
%\usepackage{titlesec}                               %章节标题样式
\usepackage{titletoc}                                   % 控制目录的宏包
\usepackage{fancyhdr}                                % 页眉和页脚的相关定义
\usepackage{caption2}                                % 浮动图形和表格标题样式
\usepackage{setspace}                                % 定制表格和图形的多行标题行距
\usepackage[perpage]{footmisc}                % 脚注控制
%---------------------------------------------------------------------------------
\usepackage{etex}                          %定义更多的计数器,否则警告"no room for a new \count"
\usepackage{graphicx}                   %插图
%\usepackage{subfigure}
%\usepackage{picins}                      %图文混排(文包图),可带方框
%\usepackage{picinpar}                  %图文混排(文包图)  picins在前,picinpar在后,否则有问题
%-------------------------------------------------------------------------------------------
\usepackage{colortbl}                      % 单元格加背景
\usepackage{multirow}                    % 制作表格主要用到\multicolumn,\multirow和\cline,其中,要使用\multirow
\usepackage{everb}                          % 扩展了verbatim,可以加边框、前景色,背景色和行号等  \begin{colorboxed}这里的内容可以跨页\end{colorboxed}
\usepackage{fancybox}                    % 边框,有阴影,fancybox提供了五种式样\fbox,\shadowbox,\doublebox,\ovalbox,\Ovalbox。
\usepackage[listings,theorems]{tcolorbox}                % 用于定义类似于beamer的环境框
\usepackage{setspace}                     % 行间距的宏包
\usepackage{extarrows}                  %实 现长等号等  This package provides some extensible arrows in maths. mode    (more than \xlongequal \xleftarrow \xrightarrow \xLongleftarrow \xLongrightarrow \xLongleftrightarrow
%\usepackage{indentfirst}               % 首行缩进宏包
%\usepackage{layouts}                    % 打印当前页面格式的宏包
%\usepackage{color}                        % 支持彩色
%\usepackage{overpic}                    % 加背景图等
%\usepackage{multicol}                  % 用于实现在同一页中实现不同的分栏
\usepackage{listings}                    % 可对关键词、注释和字符串等使用不同的字体和颜色,也可以为代码添加边框、背景等风格
%\usepackage{flushend}                 % 分两栏后,\flushend 命令使双栏底部对齐,\raggedend 命令则取消底部对齐,文档最后有效
%\usepackage{balance}                  % 分两栏后,\balance  命令使双栏底部对齐,\balance   命令则取消底部对齐,任何位置有效
% -------------------------------字体等-----------------------------------------------------------
\usepackage{times}                       % 使用Times字体的宏包
\usepackage{amssymb}                   % AMSLaTeX宏包 用来排出更加漂亮的公式
\usepackage{wasysym}                    %积分号直立
\usepackage{mathrsfs}                    % 不同于\mathcal or \mathfrak 之类的英文花体字体
\usepackage{bm}                              % 处理数学公式中的黑斜体的宏包
\usepackage[amsmath,thmmarks]{ntheorem}         % 定理类环境宏包,其中 amsmath 选项用来兼容 AMS LaTeX 的宏包
%----------------------------------------------------------------------------------------------------------------------------------------
%\usepackage{flafter}                     % 因为图形可浮动到当前页的顶部,所以它可能会出现在它所在文本的前面. 要防止这种情况,可使用 flafter 宏包
%\usepackage{endfloat}                %可将浮动对象放置到文件的最后
%\usepackage[below]{placeins}
%\usepackage{floatflt}                    % 图文混排用宏包
%\usepackage{rotating}                 % 图形和表格的控制
%\usepackage{type1cm}                % tex1cm宏包,控制字体的大小
%--------------------------------------------------------------------------------------------------
%\usepackage{fancyref}                    % 支持引用的宏包
%\usepackage{cite}                        % 支持引用缩写的宏包
%\usepackage{overcite}
\usepackage[numbers,sort&compress,super,square]{natbib}
%--------------------------------------------------------------
