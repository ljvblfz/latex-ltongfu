%-----------------------------------------------------主文档 格式定义---------------------------------
\addtolength{\headsep}{-0.1cm}        %页眉位置
%\addtolength{\footskip}{0.4cm}       %页脚位置
%===========================================================================
%\setlength{\mathindent}{4.7 em}     %左对齐公式缩进量
\allowdisplaybreaks[4]
%===========================================================================
%下面这组命令使浮动对象的缺省值稍微宽松一点,从而防止幅度
%对象占据过多的文本页面,也可以防止在很大空白的浮动页上放置
%很小的图形。
%===========================================================================
\renewcommand{\textfraction}{0.15}
\renewcommand{\topfraction}{0.85}
\renewcommand{\bottomfraction}{0.65}
\renewcommand{\floatpagefraction}{0.60}
%===========================================================================
%下面这组命令可以使公式编号随着每开始新的一节而重新开始。
%===========================================================================
%\makeatletter      % '@' is now a normail "letter" for TeX
%\@addtoreset{eqation}{section}
%\makeatother       % '@' is restored as a "non-letter" character for TeX
%-------------------------------------------------------------------------------------------
\defaultfontfeatures{Mapping=tex-text} %为了使用TEX风格的连字符
\XeTeXlinebreaklocale "zh"                      % 中文断行规则
\XeTeXlinebreakskip = 0pt plus 1pt        % 中文断行规则

%-----------------------------------------------------设定字体等------------------------------
\setmainfont{Times New Roman}             % 英文 New Roman 体,注释这一行可使全文数字用CM体
%\setCJKmainfont{SimSun}                       % 设置缺省中文字体
%\setCJKmonofont{Microsoft YaHei}       % 设置等宽字体
%\setmonofont{Courier New}                   % 英文等宽字体
%\setsansfont{Courier New}                     % 英文无衬线字体

\newCJKfontfamily\song{SimSun}
\newCJKfontfamily\hei{SimHei}
\newCJKfontfamily\kai{KaiTi}
\newCJKfontfamily\fs{FangSong}
\newCJKfontfamily\li{LiSu}
\newCJKfontfamily\you{YouYuan}
\newCJKfontfamily\yahei{Microsoft YaHei}
\newCJKfontfamily\xingkai{STXingkai}
\newCJKfontfamily\xinwei{STXinwei}
\newCJKfontfamily\fzyao{FZYaoTi}
\newCJKfontfamily\fzshu{FZShuTi}
%-----------------------------------------------------------定义颜色---------------
\definecolor{blueblack}{cmyk}{0,0,0,0.35}%浅黑
\definecolor{darkblue}{cmyk}{1,0,0,0}%纯蓝
\definecolor{lightblue}{cmyk}{0.15,0,0,0}%浅蓝
%--------------------------------------------------------设定标题颜色--------------
\CTEXsetup[format+={\color{darkblue}}]{chapter}
\CTEXsetup[format+={\color{darkblue}}]{section}
\CTEXsetup[format+={\color{darkblue}}]{subsection}
%-----------------------------------------------------------定义、定理环境-------------------------
%\theoremseparator{~}                  %帽号 : 改为 ~ 变空格
\newcounter{myDefinition}[chapter]\def\themyDefinition{\thechapter.\arabic{myDefinition}}
\newcounter{myTheorem}[chapter]\def\themyTheorem{\thechapter.\arabic{myTheorem}}
\newcounter{myCorollary}[chapter]\def\themyCorollary{\thechapter.\arabic{myCorollary}}

\newtheorem{definition}{\indent\hei\color{white}定义}[chapter]

\tcbmaketheorem{defi}{定义}{fonttitle=\bfseries\upshape, fontupper=\slshape, arc=0mm, colback=lightblue,colframe=darkblue}{myDefinition}{Definition}
\tcbmaketheorem{theo}{定理}{fonttitle=\bfseries\upshape, fontupper=\slshape, arc=0mm, colback=lightblue,colframe=darkblue}{myTheorem}{Theorem}
\tcbmaketheorem{coro}{推论}{fonttitle=\bfseries\upshape, fontupper=\slshape, arc=0mm, colback=lightblue,colframe=darkblue}{myCorollary}{Corollary}
%------------------------------------------------------------------------------

\theoremheaderfont{\hei\rmfamily}\theorembodyfont{\rmfamily}
\newtheorem{exam}{\indent\hei \color{darkblue}例}[chapter]
\newtheorem{example}{\hspace{2em}\hei\color{white}Matlab程序}[chapter]
\newcommand{\program}[1]{\noindent\parbox{14.6cm}{\includegraphics[width=3.2cm]{fig/program}
\vspace{-1.12cm}\example\textcolor{darkblue}{\hei#1}}\vspace{0.0cm}}


\newtheorem{proof}{\indent\hei \textcolor{darkblue}{证明}}
\newtheorem{Solution}{\indent\hei \textcolor{darkblue}{解}}
\newcommand{\zhijia}{\vphantom{\bigg|}} % 行内公式用的垂直支架
\newcommand{\zhijiaa}{\vphantom{\Bigg|}} % 行内公式用的垂直支架
\setlength{\parindent}{1.7em}    %首行缩进2个汉字

%==========定义框====================================
\newcommand{\definibox}[2]{\vspace{0.3cm}
\begin{colorboxed}[boxcolor=darkblue,bgcolor=darkblue]
\vspace{-0.4cm}
\definition \hei\color{white}#1
\vspace{0.04cm}
\end{colorboxed}
\vspace{-0.7cm} \setlength{\fboxsep}{4pt}
\begin{colorboxed}[boxcolor=darkblue,bgcolor=lightblue]
#2
\end{colorboxed}
\color{black}
}
%==========定义框结束====================================

%===============定义章节标题样式================
\setcounter{secnumdepth}{4}
\setcounter{tocdepth}{4}
%-----------------------------------设定标题颜色---------------------------
\CTEXsetup[format+={\color{darkblue}}]{chapter}
\CTEXsetup[format+={\color{darkblue}}]{section}
\CTEXsetup[format+={\color{darkblue}}]{subsection}

\renewcommand\contentsname{Contents}
\renewcommand\bibname{References}
%-----------------------------------使用titlesec宏包设定标题颜色---------------------------
%\titleformat{\chapter}[hang]{\centering\color{darkblue}\zihao{2}\bfseries}{第\chaptername 章 }{20pt}{}
%\titlespacing{\chapter}{0pt}{50pt}{40pt}                                          % 标题左边距,上文距,下文距
%\titleformat{\section}{\centering\color{darkblue}\zihao{4}\bfseries}{\S\,\thesection}{1em}{}
%\titlespacing{\section}{0pt}{3.5ex plus 1ex minus .2ex}{2.3ex plus .2ex}         % 标题左边距,上文距,下文距
%\titleformat{\subsection}{\color{darkblue}\zihao{-4}\bfseries}{\thesubsection}{1em}{}
%\titlespacing*{\subsection} {0pt}{3.25ex plus 1ex minus .2ex}{1.5ex plus .2ex}   % 标题左边距,上文距,下文距
%去掉中间对齐的sectionformat,这样就把节的标题左对齐了。
%-------------------------------------------------------------------------------------------------------------------------------
% 按清华标准, 缩小目录中各级标题之间的缩进
\dottedcontents{chapter}[0.0em]{\hei\vspace{0.5em}}{0.0em}{5pt}
\dottedcontents{section}[1.16cm]{}{1.8em}{5pt}
\dottedcontents{subsection}[2.00cm]{}{2.7em}{5pt}
\dottedcontents{subsubsection}[2.86cm]{}{3.4em}{5pt}

%================== 定义页眉和页脚使用fancyhdr 宏包===================================

%------------------------------------------------定义页眉下单隔线----------------
\newcommand{\makeheadrule}{\makebox[0pt][l]{\color{darkblue}\rule[.7\baselineskip]{\headwidth}{0.3pt}}\vskip-.8\baselineskip}
%-----------------------------------------------定义页眉下双隔线----------------
%\newcommand{\makeheadrule}{%
%    \makebox[0pt][l]{\color{darkblue}\rule[.7\baselineskip]{\headwidth}{0.3pt}}%0.4
%    \color{darkblue}\rule[0.85\baselineskip]{\headwidth}{1.5pt}\vskip-.8\baselineskip}%1.5 0.4->0.5
\makeatletter
\renewcommand{\headrule}{{\if@fancyplain\let\headrulewidth\plainheadrulewidth\fi\makeheadrule}}
\pagestyle{fancy}
\renewcommand{\chaptermark}[1]{\markboth{第\chaptername 章\quad #1}{}}    %去掉章标题中的数字
\renewcommand{\sectionmark}[1]{\markright{\thesection\quad #1}{}}    %去掉节标题中的点
\fancyhf{} %清空页眉

%\lhead{\resizebox{70pt}{!}{\includegraphics{fig/renmindaxue.png}}}   %页眉插图
%\rhead{\resizebox{70pt}{!}{\includegraphics{fig/renmindaxue.png}}}   %页眉插图

%在 book文件类别下,\leftmark自动存录各章之章名,\rightmark记录节标题
%\fancyhead[RO]{\kai{\wuhao.~\color{darkblue}\thepage~.}\hspace{-2em}}         % 奇数页码显示左边
%\fancyhead[LE]{\hspace{-2em}\kai{\wuhao.~\color{darkblue}\thepage~.}}         % 偶数页码显示右边

\fancyhead[RO]{\kai{\footnotesize.~\color{darkblue}\thepage~.}}         % 奇数页码显示左边
\fancyhead[LE]{\kai{\footnotesize.~\color{darkblue}\thepage~.}}         % 偶数页码显示右边
\fancyhead[CO]{\song\footnotesize\color{darkblue}\rightmark} % 奇数页码中间显示节标题
\fancyhead[CE]{\song\footnotesize\color{darkblue}\leftmark}  % 偶数页码中间显示章标题

%\fancyhead[CO]{\mbox{~~~~}\hfill\makebox{\CJKfamily{song}\wuhao\rightmark}\hfill\mbox{\wuhao.~\thepage~.}}
%\fancyhead[CE]{\mbox{\wuhao.~\thepage~.}\hfill\makebox{\CJKfamily{song}\wuhao\leftmark}\hfill\mbox{~~~~}}
%\fancyfoot[C,C]{\wuhao--~\thepage~--}          % 页码在版芯下边线之下隔行居中放置;

%=========================================================
\setlength{\parskip}{3pt plus1pt minus1pt}      % 段落之间的竖直距离
\renewcommand{\baselinestretch}{1.25}           % 定义行距
%========================调整列表环境的垂直间距==================

\let\orig@Itemize =\itemize
\let\orig@Enumerate =\enumerate
\let\orig@Description =\description
\def\Myspacing{\itemsep=1.5ex \topsep=-0.5ex \partopsep=0pt \parskip=0pt \parsep=0.5ex}
\def\newitemsep{
\renewenvironment{itemize}{\orig@Itemize\Myspacing}{\endlist}
\renewenvironment{enumerate}{\orig@Enumerate\Myspacing}{\endlist}
\renewenvironment{description}{\orig@Description\Myspacing}{\endlist}
}
\def\olditemsep{
\renewenvironment{itemize}{\orig@Itemize}{\endlist}
\renewenvironment{enumerate}{\orig@Enumerate}{\endlist}
\renewenvironment{description}{\orig@Description}{\endlist}
}
\newitemsep

%================================调整公式与正文间距=========================
\renewcommand\normalsize{%
   \@setfontsize\normalsize\@xpt\@xiipt
   \abovedisplayskip 5\p@ \@plus3\p@ \@minus5\p@
   \abovedisplayshortskip \z@ \@plus3\p@
   \belowdisplayshortskip 5\p@ \@plus3\p@ \@minus3\p@
   \belowdisplayskip\abovedisplayskip
   \let\@listi\@listI}
\setlength\jot{10pt}  %多行公式内部间距

%\setlength\abovedisplayskip{0pt}
%\setlength\belowdisplayskip{0pt}
%renewcommand{\arraystretch}{1.5}

%\abovedisplayskip=12pt plus 3pt minus 9pt
%\abovedisplayshortskip=0pt plus 3pt
%\belowdisplayskip=12pt plus 3pt minus 9pt
%\belowdisplayshortskip=7pt plus 3pt minus 4pt

%==============================调整表格标题与正文及表格间距===========================
\setlength{\abovecaptionskip}{0.1cm}
\setlength{\belowcaptionskip}{0.1cm}
%===========================修改引用的格式==============================
%第一行在引用处数字两边加方框
%第二行去除参考文献里数字两边的方框
%\makeatletter
%\def\@cite#1{\mbox{$\m@th^{\hbox{\@ove@rcfont[#1]}}$}}
%\renewcommand\@biblabel[1]{#1}
%\makeatother
% 增加 \ucite 命令使显示的引用为上标形式
\newcommand{\ucite}[1]{$^{\mbox{\scriptsize \cite{#1}}}$}
%==============================定制浮动图形和表格标题样式===========================
\renewcommand{\captionfont}{\CJKfamily{song}\rmfamily}
\renewcommand{\captionlabelfont}{\CJKfamily{song}\rmfamily}
%=========================================================
%  去掉图表号后面的:
%=========================================================
%\renewcommand{\captionlabeldelim}{\hspace{1em}}
%=========================================================
%  图表标题字体为10pt, 这里写作小五号
%=========================================================
\renewcommand{\captionfont}{\footnotesize}
%\renewcommand{\captionfont}{\footnotesize\bf}   %加黑
%=============================封面============================
\def\cftitle#1{\def\@cftitle{#1}}\def\@cftitle{}
\def\ctitle#1{\def\@ctitle{#1}}\def\@ctitle{}
\def\cauthor#1{\def\@cauthor{#1}}\def\@cauthor{}
\def\publisher#1{\def\@publisher{#1}}\def\@publisher{}
\def\makecover{
    \begin{titlepage}
    \begin{center}
    \parbox[t][2.0cm][c]{\textwidth}{\zihao{3} \begin{center} {\song \@cftitle}\end{center} }
%    \parbox[t][4cm][c]{\textwidth}{\linghao\begin{center} {\hei \@ctitle}\end{center} }
    \parbox[t][0.0cm][c]{\textwidth}{\zihao{0}\begin{center} {\hei \@ctitle}\end{center} }
    \parbox[t][4.0cm][c]{\textwidth}{\zihao{3}\begin{center} {\song \@cauthor}\end{center} }
    \parbox[t][11.5cm][b]{\textwidth}{%\centering \includegraphics[totalheight=1.0in]{fig/qinghua.jpg} \\
                    \zihao{3}\begin{center} { \kai \@publisher} \end{center}}
    \end{center}
    \end{titlepage}
    \normalsize
    }
%=============================封面结束============================
