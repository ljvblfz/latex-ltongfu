%-----------------------------------------------------主文档 格式定义---------------------------------
\addtolength{\headsep}{-0.1cm}        %页眉位置
%\addtolength{\footskip}{0.4cm}       %页脚位置
%-----------------------------------------------------设定字体等------------------------------
\setmainfont{Times New Roman}    % 缺省字体
\setCJKfamilyfont{song}{SimSun}
\setCJKfamilyfont{hei}{SimHei}
\setCJKfamilyfont{kai}{KaiTi}
\setCJKfamilyfont{fs}{FangSong}
\setCJKfamilyfont{li}{LiSu}
\setCJKfamilyfont{you}{YouYuan}
\setCJKfamilyfont{yahei}{Microsoft YaHei}
\setCJKfamilyfont{xingkai}{STXingkai}
\setCJKfamilyfont{xinwei}{STXinwei}
\setCJKfamilyfont{fzyao}{FZYaoTi}
\setCJKfamilyfont{fzshu}{FZShuTi}
%-------------------------------------------------------------------
\newCJKfontfamily\song{SimSun}
\newCJKfontfamily\hei{SimHei}
\newCJKfontfamily\kai{KaiTi}
\newCJKfontfamily\fs{FangSong}
\newCJKfontfamily\li{LiSu}
\newCJKfontfamily\you{YouYuan}
\newCJKfontfamily\yahei{Microsoft YaHei}
\newCJKfontfamily\xingkai{STXingkai}
\newCJKfontfamily\xinwei{STXinwei}
\newCJKfontfamily\fzyao{FZYaoTi}
\newCJKfontfamily\fzshu{FZShuTi}
%-----------------------------------------------------------定义颜色---------------
\definecolor{blueblack}{cmyk}{0,0,0,0.35}%浅黑
\definecolor{darkblue}{cmyk}{1,0,0,0}%纯蓝
\definecolor{lightblue}{cmyk}{0.15,0,0,0}%浅蓝
%--------------------------------------------------------设定标题颜色--------------
\CTEXsetup[format+={\color{darkblue}}]{chapter}
\CTEXsetup[format+={\color{darkblue}}]{section}
\CTEXsetup[format+={\color{darkblue}}]{subsection}
%-----------------------------------------------------------定义、定理环境-------------------------
\newcounter{myDefinition}[chapter]\def\themyDefinition{\thechapter.\arabic{myDefinition}}
\newcounter{myTheorem}[chapter]\def\themyTheorem{\thechapter.\arabic{myTheorem}}
\newcounter{myCorollary}[chapter]\def\themyCorollary{\thechapter.\arabic{myCorollary}}

\tcbmaketheorem{defi}{定义}{fonttitle=\bfseries\upshape, fontupper=\slshape, arc=0mm, colback=lightblue,colframe=darkblue}{myDefinition}{Definition}
\tcbmaketheorem{theo}{定理}{fonttitle=\bfseries\upshape, fontupper=\slshape, arc=0mm, colback=lightblue,colframe=darkblue}{myTheorem}{Theorem}
\tcbmaketheorem{coro}{推论}{fonttitle=\bfseries\upshape, fontupper=\slshape, arc=0mm, colback=lightblue,colframe=darkblue}{myCorollary}{Corollary}
%------------------------------------------------------------------------------
\newtheorem{proof}{\indent\hei \textcolor{darkblue}{证明}}
\newtheorem{Solution}{\indent\hei \textcolor{darkblue}{解}}
%------------------------------------------------定义页眉下单隔线----------------
\newcommand{\makeheadrule}{\makebox[0pt][l]{\color{darkblue}\rule[.7\baselineskip]{\headwidth}{0.3pt}}\vskip-.8\baselineskip}
%-----------------------------------------------定义页眉下双隔线----------------
\makeatletter
\renewcommand{\headrule}{{\if@fancyplain\let\headrulewidth\plainheadrulewidth\fi\makeheadrule}}
\pagestyle{fancy}
\renewcommand{\chaptermark}[1]{\markboth{第\chaptername 章\quad #1}{}}    %去掉章标题中的数字
\renewcommand{\sectionmark}[1]{\markright{\thesection\quad #1}{}}    %去掉节标题中的点
\fancyhf{} %清空页眉
\fancyhead[RO]{\kai{\footnotesize.~\color{darkblue}\thepage~.}}         % 奇数页码显示左边
\fancyhead[LE]{\kai{\footnotesize.~\color{darkblue}\thepage~.}}         % 偶数页码显示右边
\fancyhead[CO]{\song\footnotesize\color{darkblue}\rightmark} % 奇数页码中间显示节标题
\fancyhead[CE]{\song\footnotesize\color{darkblue}\leftmark}  % 偶数页码中间显示章标题
%---------------------------------------------------------------------------------------------------------------------


